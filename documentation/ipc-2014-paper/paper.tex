%%% AAAI
\documentclass[letterpaper]{article}

\usepackage{aaai}
\usepackage{times}
\usepackage{helvet}
\usepackage{courier}
\usepackage{url}
\usepackage{booktabs}
\usepackage{graphics}

\begin{document}

\title{Width and Inference Based Planners: SIW, BFS(f), and PROBE}

% \author{Name1 Surname1 \and Name2 Surname2 \and Name3 Surname3\institute{University of Leipzig,
% Germany, email: somename@informatik.uni-leipzig.de} }

%%% AAAI
\author{Nir Lipovetzky \\
      University of Melbourne \\
   Melbourne, Australia\\ 
   {\normalsize\url{@unimelb.edu.au}}
   \And
   Miquel Ramirez \\
   RMIT University \\
   Melbourne, Australia\\ 
   {\normalsize\url{@rmit.edu.au}}
   \And
   Christian Muise \\
   University of Melbourne \\
   Melbourne, Australia\\ 
   {\normalsize\url{@unimelb.edu.au}}
   \And
        Hector Geffner \\
        ICREA \&  Universitat Pompeu Fabra \\
        Barcelona, SPAIN \\
        {\normalsize\url{@upf.edu}}\footnote{firstname.lastname}
}

\maketitle


\newcommand{\tuple}[1]{{\langle #1\rangle}}
\newcommand{\triple}[1]{{\langle #1\rangle}}
\newcommand{\pair}[1]{{\langle #1\rangle}}

\newcommand{\Omit}[1]{}

\newcommand{\OmitEcai}[1]{}

\newcommand{\eqdef}{\stackrel{\hbox{\tiny{def}}}{=}}
% \newcommand{\IR}{{\textit{IR}}}
% \newcommand{\SR}{{\textit{SR}}}
\newcommand{\IR}{{\textit{IW}}}
\newcommand{\SR}{{\textit{SIW}}}
\newcommand{\ID}{{\textit{ID}}}

\newcommand{\BRFS}{{\textit{BrFS}}}

 \newtheorem{theorem}{Theorem}


\newtheorem{proposition}[theorem]{Proposition}
\newtheorem{lemma}[theorem]{Lemma}
\newtheorem{corollary}[theorem]{Corollary}
\newtheorem{definition}[theorem]{Definition}



\maketitle
\Omit{
  \begin{abstract}
    \begin{quote}

      We have recently shown that classical planning problems can be
      characterized in terms of a width measure that is bounded and small
      for most planning benchmark domains when goals are restricted to
      single atoms. Two simple algorithms have been devised for exploiting
      this structure: Iterated Width (IW) for achieving atomic goals, that
      runs in time exponential in the problem width by performing a sequence
      of pruned breadth first searches, and Serialized IW (SIW) that uses IW
      in a greedy search for achieving conjunctive goals one goal at a time.
      While SIW does not use heuristic estimators of any sort, it manages to
      solve more problemas that a Greedy BFS with heuristics like
      $h_{add}$. Yet, it does not approach the peformance of more recent
      planners like LAMA. In addition, ideas like helpful actions and
      landmarks can be integrated as well, producing a planner with
      state-of-the-art performance.
    \end{quote}
  \end{abstract}
}
% AAAI
\section{Introduction}

\Omit{
  The main approach for domain independent planning is based on 
  heuristic search with heuristics derived automatically from problems
  \cite{mcdermott:unpop,bonet:aij-hsp}. To this, recent planners 
  add other ideas like helpful actions,  landmarks, and
  multiqueue best-first  search for  combining  different heuristics
  \cite{hoffmann:ff,helmert:fd,richter:lama:aij}. 
  From a different angle, we have recently shown that most of the benchmark 
  domains are easy when the goals contain single atoms,
  and that otherwise, they can be easily serialized
  \cite{nir:ecai12}. More precisely, we showed that the former
  problems have a low width, and developed  an algorithm, 
  Iterative Width (IW) that runs in time that
  is exponential in the problem width by performing
  a sequence of pruned breadth first searches. 
  This  algorithm is used in the context of another algorithm,
  Serialized IW (SIW),  that achieves conjunctive goals
  by using IW greedily for achieving one goal at a time.
  Surprisingnly,  the blind-search algorithm SIW
  which has no heuristic guidance of any sort, performs better
  than a greedy best-first search guided by delete-relaxation
  heuristics.
}

We have entered the planners $\SR$, $BFS(f)$, and $PROBE$ to the
\emph{agile-track} of the 2014 International Planning Competition, and an anytime
planner for the \emph{satisficing track} that runs both $\SR$ and $BFS(f)$.
SIW and BFS(f) are planners that make use of width-considerations \cite{nir:ecai12}, while PROBE is a standard best-first search planner that before expanding a node, it throws a probe  that
either reaches the goal or terminates in low polynomial time~\cite{nir:icaps11}.

The basic building block of $\SR$ is the Iterative Width Procedure
($\IR$) for achieving atomic goals, that runs in time exponential in
the problem width by performing a sequence of pruned breadth first
searches.  The planner $BFS(f)$ integrates a \emph{novelty} measure
borrowed from $\IR$ with helpful-actions, landmarks and
delete-relaxation heuristics in a Greedy Best-Fist
search.  

 All planners are based in a new toolkit for automated planning
 \cite{} aiming at [.....]



In the following sections we introduce the basic notions of the
algortihms and the implementation. We assume a STRIPS problem $P =
\langle F,I,O,G\rangle$, where $F$ is the set of atoms, $I$ is the set
of atoms characterizing the initial state, $O$ is the set of actions,
and $G$ is the set of goal atoms.




% AAAI
\section{Iterated Width Search}


The algorithm  \emph{Iterated Width}  or \IR\ 
consists  of a sequence of calls $\IR(i)$ for $i=0,1,\ldots, |F|$
until the problem is solved. Each iteration $\IR(i)$ is a 
breadth-first search that prunes right away  states  that do not pass a   \emph{novelty} test; 
namely, for a state $s$ in $\IR(i)$ \emph{not} to be pruned there must be 
a tuple $t$ of at most $i$ atoms such that $s$ is the first state 
generated in the search that makes $t$ true. The time complexities of $\IR(i)$ and $\IR$
are  $O(n^i)$ and $O(n^w)$ respectively  where $n$ is $|F|$ and $w$ is the problem width.
The width of existing domains is low for atomic goals,  and indeed,  89\% of the  benchmarks
can be solved by $\IR(2)$ when the goal is set to any  of the atoms in the goal \cite{nir:ecai12}.
The width of the benchmark domains with conjunctive goals, however, is not low in general,
yet such problems can be serialized.


The algorithm \emph{Serialized Iterative width} or $\SR$ uses $\IR$
for serializing a problem into subproblems and for solving the
subproblems.  Basically, SIW uses IW to achieve one atomic goal at a
time, greedily, until all atomic goals are achieved jointly. In
between, atomic goals may be undone, but after each invocation of IW,
each of the previously achieved goals must hold. SIW will thus never
call IW more than $|G|$ times where $|G|$ is the number of atomic
goals. SIW compares surprisingly well to a baseline heuristic search
planner based on greedy best-first search and the $h_{add}$
heuristic~\cite{bonet:aij-hsp}, but doesn't approach the performance
of the most recent planners. Still, $\SR$ is able to compete well in
domains that accept simple serializations and the induced problems can
be solved by $\IR(i)$ with $i=1,\ldots,3$.

\Omit{

  The  algorithm, called  \emph{Iterated Width} search or \IR, 
  consists of a sequence of calls  $\IR(i)$ for $i=0, 1, 2, \ldots$
  over a problem $P$ until the problem is solved.
  Each iteration $\IR(i)$ is an $i$-width search that is complete for problems whose width is bounded by $i$
  and whose  complexity is $O(n^{i})$,  where $n$ is the number of problem variables.
  If $P$ is solvable and its  width is $w$,  \IR\ will solve $P$ in at most $w+1$ iterations
  with a   complexity  $O(n^w)$.  $\IR(i)$ is 
  \emph{a plain forward-state breadth-first search} with just one change: right after a state $s$ is generated, 
  the state is  pruned if it doesn't pass a simple \emph{novelty test} that depends on $i$.

  \begin{definition}
    A newly generated state $s$  \emph{generates a  new tuple of atoms} $t$
    iff $s$ is the first state generated in the search  that makes $t$ true. The size of the smallest new tuple
    of atoms generated by $s$ is called the \emph{novelty} of $s$. When $s$ does not generate a new
    tuple, it's novelty is set to $n+1$ where $n$ is the number of problem variables.
  \end{definition}

  In other words, if  $s$ is the first state generated in all the search that makes an atom $p$ true,
  its novelty is  $1$. If $s$ does not generate a new atom but generates a new pair $(p,q)$,
  its novelty is $2$, and so on.  Likewise, if $s$ does not generate a new tuple at all
  because the same state has been generated before, then its novelty is set to $n+1$.
  The higher the  novelty measure, the less novel the state.
  The iterations $\IR(i)$ are  plain \emph{breadth-first searches} that treat newly generated
  states with novelty measure greater than $i$ as if they were  `duplicate' states:


  \begin{definition}
    $\IR(i)$ is a breadth-first search that prunes newly generated states when their novelty measure
    is greater than $i$.
  \end{definition}

  Notice that $\IR(n)$, when $n$ is the number of atoms in the problem, just prunes truly duplicate states
  and it  is therefore complete.
  On the other hand, $\IR(i)$ for lower $i$ values prunes many states and is not.
  Indeed, the number of states \emph{not} pruned in $\IR(1)$  is $O(n)$
  and similarly, the number of states not pruned in $\IR(i)$ is $O(n^i)$.
  Likewise, since the novelty of a state is never $0$, $\IR(0)$ prunes all the children
  states of the initial state $s_0$, and thus $\IR(0)$  solves  $P$ iff
  the goal is true in the initial situation.  The resulting planning algorithm $\IR$
  is just a series of $i$-width searches $\IR(i)$, for increasing values of $i$:

  \begin{definition}
    Iterated Width  (\IR) calls $\IR(i)$ sequentially for $i=0,1,2,\ldots$ until
    the problem is solved or $i$ exceeds the number of problem variables.
  \end{definition}



  Iterated  Width (\IR) is thus a \emph{blind-search algorithm} similar to Iterative Deepening (\ID)
  except for two differences. First, each iteration  is a pruned \emph{depth-first} search in \ID,
  and  a pruned \emph{breadth-first} search in \IR. Second,  each iteration increases pruning  depth
  in \ID, and pruning width or novelty in \IR.


  From the considerations above it is straightforward to show that
  \IR\ like \ID\ is \emph{sound} and \emph{complete.}  On the other
  hand, while $\IR(w)$ is \emph{optimal} for a problem $P$ of width $w$,
  $\IR$ is not necessarily so. The reason is that $\IR$ may solve $P$ in
  an iteration $\IR(i)$ for $i$ smaller than $w$.%%%%


  Nonetheless the completeness and optimality of $\IR(w)$ for problems
  with width $w$ provides the right complexity bound for $\IR$:
  %
  \begin{theorem}
    For solvable problems $P$, the time and space complexity of $\IR$
    are  exponential in $w(P)$.
  \end{theorem}
  %
  It's important to realize that this bound is achieved without knowing the
  actual width of $P$. This follows from the result below, whose
  proof we omit for lack of space:

  \begin{theorem}
    For a  solvable problem $P$ with width $w$,  $\IR(w)$ solves $P$ optimally
    in time exponential in $w$.
  \end{theorem}



  The  algorithm $\IR(w)$ is guaranteed to solve $P$ if $w(P)=w$,
  yet as discussed above, the algorithm $\IR$ does not assume
  that this width is known  and thus makes the $\IR(i)$ calls in order
  starting from $i=0$. We  refer to the min value of $i$
  for which $\IR(i)$ solves $P$ as the \emph{effective width} of $P$,
  $w_e(P)$, which is never higher than the real width $w(P)$.



  The effective width $w_e(P)$ provides an approximation of the actual
  width $w(P)$.  While proving formally that most benchmark domains have
  bounded width \emph{for single atom goals} is tedious, we have run the
  algorithm $\IR$ to compute the \emph{effective width} of such
  goals. Over all domains of Previous IPC: $37\%$ with $w_e=1$, $51\%$
  with $w_e=2$, and less than $12\%$ with $w_e > 2$. \emph{That is, on
    average, less than $12\%$ of the instances have effective width
    greater than $2$.}  Actually, in most domains \emph{all} the
  instances have effective width at most $2$, and in four domains, all
  the instances have effective width $1$. We will see below that a
  simple extension suffices to make \IR\ competitive with a
  \emph{heuristic} planner over the standard benchmark instances that
  feature \emph{joint goals}.

  % AAAI
  \section{Serialized Iterated Width}


  The fact that single goal atoms can be achieved quite effectively in
  most benchmarks domains by a pruned breadth-first search that does not
  look at the goal in any way, suggests that the complexity of
  benchmarks comes from conjunctive goals. Indeed, this has been the
  intuition in the field of planning since its beginnings where goal
  decomposition was deemed as a crucial and characteristic
  technique. The analysis above formalizes this intuition by showing
  that the effective width of single atom goals in existing benchmarks
  is low.  This old intuition also suggests that the power of planners
  that can handle single goals efficiently can be exploited for
  conjunctive goals through some form of decomposition.


  \emph{Serialized Iterated Width} is a search algorithm that uses the
  iterated width searches both for constructing a serialization of the
  problem $P = \langle F,I,O,G\rangle$ and for solving the resulting
  subproblems. While $\IR$ is a sequence of $i$-width searches $\IR(i)$,
  $i=0,1,\ldots$ over the same problem $P$, $\SR$ is a sequence of $\IR$
  calls over $|G|$ subproblems $P_k$, $k=1, \ldots, |G|$.  The
  definition of $\SR$ takes advantage of the fact that $\IR$ is a
  blind-search procedure that doesn't need to know the goals of the
  problem in advance; it just needs to recognize them in order to stop.
  Thanks to this feature $\IR$ is used both for decomposing $P$ into the
  sequence of subproblems $P_k$ and for solving each one of them. The
  plan for $P$ is the concatenation of the plans obtained for the
  subproblems.

  \begin{definition}
    Serialized Iterated Width (\SR) over $P=\langle F,I,O,G\rangle$
    consists of a sequence of calls to $\IR$ over the problems $P_k
    =\langle F,I_k,O,G_k\rangle$, $k=1, \ldots, |G|$, where

    \begin{enumerate}
    \item $I_1=I$,
    \item $G_{k}$ is first \emph{consistent}  set of atoms  achieved from $I_k$
      such that $G_{k-1} \subset G_k \subseteq G$ and $|G_k|=k$; $G_0 = \emptyset$
    \item $I_{k+1}$ represents the   state where  $G_k$ is  achieved, $1 < k < |G|$.
    \end{enumerate}
  \end{definition}


  In other words, the $k$-th subcall of $\SR$ stops when $\IR$ generates
  a state $s_k$ that consistently achieves $k$ goals from $G$: those
  achieved in the previous subcall and a new goal from $G$. The same is
  required from the next subcall that starts at $s_k$.  The state $s_k$
  \emph{consistently} achieves $G_k \subseteq G$ if $s_k$ achieves
  $G_k$, and $G_k$ does not need to be undone in order to achieve
  $G$. This last condition is checked by testing whether
  $h_{max}(s_k)=\infty$ is true in $P$ once the actions that delete
  atoms from $G_k$ are excluded \cite{bonet:aij-hsp}.  Notice that $\SR$
  does not use heuristic estimators to the goal, and does not even know
  what goal $G_k$ is when $\IR$ is invoked on subproblem $P_k$: it finds
  this out when $\IR$ generates a set of atoms $G'$ such that $G_{k-1}
  \subset G' \subseteq G$ and $|G'|=k$. It then sets $G_k$ to $G'$. This
  is how $\SR$ manages to use $\IR$ for both constructing the
  serialization and solving the subproblems.

  The $\SR$ algorithm is sound and the solution to $P$ can be obtained by concatenating the
  solutions to the problems $P_1$, \ldots, $P_m$, where $m = |G|$. Like $\IR$, however,
  $\SR$ does not guarantee the optimality of the plans found.
  Likewise, while the $\IR$ algorithm is complete, $\SR$ is not.  The reason is that the subgoal
  mechanism implicit in $\SR$ commits to intermediate states from which the goal may not be reachable.
  Of course, if there are no dead-ends in the problem, $\SR$ is complete.

}

%%
% AAAI
\section{Novelty Best-First Search}


While the blind-search \SR\ procedure competes well with a greedy
best-first planner using the additive heuristic, neither planner is
state-of-the-art. Since state-of-the-art performance is important in
classical planning, we show next that it is possible to deliver such
performance by integrating the idea of novelty that arises from width
considerations, with known techniques such as helpful actions,
landmarks, and heuristics. For this we switch to a \emph{plain
  forward-search best-first planner} guided by an evaluation function
$f(n)$ over the nodes $n$ given by

\begin{equation}
%  f(n) =   h_{add}(n) +  k \cdot usg(n)  + k' \cdot novelha(n) \ ,
 f(n) =   novelha(n) 
\label{f}
\end{equation}

\noindent where $novelha(n)$ is a measure that combines novelty and
helpful actions, as defined below.  In addition, ties are broken
lexicographically by two other measures: first, $usg(n)$, that counts
the number of subgoals not yet achieved up to $n$, and second,
$h_{add}(n)$, that is the additive heuristic.


The subgoals are the problem landmarks \cite{hoffmann:landmarks}\  derived using a
standard polynomial  algorithm over the delete-relaxation \cite{givan:landmarks,keyder:ecai10}.
% The term $k \cdot usg(n)$ provides a `reward'  $k$ (a discount in the evaluation function)
% when a subgoal is achieved. 

The count $usg(n)$ is similar to the landmark heuristic in LAMA \cite{richter:lama},
simplified  a bit:  we use only atomic landmarks (no disjunctions), sound orderings, 
and count a top goal $p$ as achieved when goals $q$ that must be established before
$q$ have been achieved \cite{nir:icaps11}.



The $novelha(n)$ measure combines the novelty of $n$ and whether the action leading to $n$
is helpful or not \cite{hoffmann:ff}. The novelty of $n$ is
defined as the size of the smallest tuple $t$ of atoms that is true in  $n$ and
false in  \emph{all previously generated nodes $n'$ in the search
with the  same number of unachieved goals $usg(n') = usg(n)$.}
Basically, nodes $n$ and $n'$ in the search with different number of unachieved goals,
$usg(n) \not= usg(n')$, are treated as being about different subproblems,
and are not compared for determining their novelty.
The novelty of a node $novel(n)$ is computed approximately, being set to $3$
when it's neither $1$ nor $2$. Similarly, if $help(n)$ is set  to $1$ or $2$
according to whether the action leading to $n$ was helpful or not,
then $novelha(n)$ is set to  a  number between 1 and 6 defined as

\begin{equation}
novelha(n) = 2 [novel(n)-1] + help(n) \ .
\label{novelha}
\end{equation}

\noindent That is, $novelha(n)$ is $1$ if the novelty of $n$ is $1$
and the action leading to $n$ is helpful, $2$ if the novelty is $1$
and the action is not helpful, $3$ if the novelty is $2$ and the
action is helpful, and so on. Basically, novel states (lower
$novel(n)$ measure) are preferred to less novel states, and helpful
actions are preferred to non-helpful, with the former criterion
carrying more weight. Nodes generated by non-helpful actions delay the
evaluation of $h_{adad}(n)$. Once again, the criterion is simple and
follows from performance considerations.



\section{PROBE: The Planner}

PROBE is a complete, standard greedy best first search (GBFS) STRIPS
planner using the standard {additive heuristic}~\cite{bonet:aij-hsp}
($h_{ff}$ if conditional effects are present ~\cite{hoffmann:ff}),
with just \emph{one change}: when a state is selected for expansion,
it first launches a \emph{probe} from the state to the goal. If the
probe reaches the goal, the problem is solved and the solution is
returned. Otherwise, the states expanded by probe are added to the
open list, and control returns to the GBFS loop.  \emph{The crucial
  and only novel part in the planning algorithm is the definition and
  computation of the probes}~\cite{nir:icaps11}.

PROBE is based on an early-version of the automated planning toolkit
that supports the implementation details of the width-based
algorithms. The only difference with respect to PROBE-IPC7 is that the
anytime procedure is disabled, as we are only concerned with the first
solution.

\section{Automated Planning Toolkit}
.
.
.
.
.

Miquel, Christian, here we can explain the implementation details


\section{Experiments}

We call the resulting best-first search planner, BFS($f$), and compare
it with three state-of-the-art planners: FF, LAMA, and PROBE
\cite{hoffmann:ff,richter:lama,nir:icaps11}.\footnote{FF is FF2.3,
  while PROBE and LAMA are from the 2011 IPC.}  Like LAMA, BFS($f$)
uses delayed evaluation, a technique that is useful for problems with
large branching factors \cite{richter:preferred-ops-and-deferred}.

.  .  .  .  I can generate a table with results over the last IPC
taking 5min max, (agile track) and keeping just the time score.

On the other hand, runing the 30min anytime-algorithm will take ages..., and
we should run lama-anytime as well for compare comparison. we may skip this,
unless you think it's really meaningful...

% AAAI
\section{Discussion}
.
.
.
.

\subsubsection{Acknowledgments}

\bibliographystyle{aaai}

\bibliography{thesisbib,crossref}


\end{document}

