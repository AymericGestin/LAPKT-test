
We entered the planners $\SR$, $BFS(f)$, and $PROBE$ to the
\emph{agile-track} of the 2014 International Planning Competition, and an anytime
planner for the \emph{satisficing track} that runs both $\SR$ and $BFS(f)$.
SIW and $BFS(f)$ are planners that make use of a notion of width for classical planning \cite{nir:ecai12}, while PROBE is a standard best-first search planner that augments the expansion of a node by throwing a probe which
either reaches the goal or terminates in low polynomial time~\cite{nir:icaps11}.

The basic building block of $\SR$ is the Iterative Width Procedure
($\IR$) for achieving atomic goals. $\IR$ runs in time exponential in
the problem width by performing a sequence of pruned breadth first
searches.  The planner $BFS(f)$ integrates a \emph{novelty} measure
borrowed from $\IR$ with helpful-actions, landmarks and
delete-relaxation heuristics in a Greedy Best-Fist
search.  

In the following sections we introduce the basic notions of the
algorithms and the implementation. We assume a STRIPS problem $P =
\langle F,I,O,G\rangle$, where $F$ is the set of atoms, $I$ is the set
of atoms characterizing the initial state, $O$ is the set of actions,
and $G$ is the set of goal atoms.
